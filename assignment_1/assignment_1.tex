% Options for packages loaded elsewhere
\PassOptionsToPackage{unicode}{hyperref}
\PassOptionsToPackage{hyphens}{url}
%
\documentclass[
]{article}
\usepackage{lmodern}
\usepackage{amssymb,amsmath}
\usepackage{ifxetex,ifluatex}
\ifnum 0\ifxetex 1\fi\ifluatex 1\fi=0 % if pdftex
  \usepackage[T1]{fontenc}
  \usepackage[utf8]{inputenc}
  \usepackage{textcomp} % provide euro and other symbols
\else % if luatex or xetex
  \usepackage{unicode-math}
  \defaultfontfeatures{Scale=MatchLowercase}
  \defaultfontfeatures[\rmfamily]{Ligatures=TeX,Scale=1}
\fi
% Use upquote if available, for straight quotes in verbatim environments
\IfFileExists{upquote.sty}{\usepackage{upquote}}{}
\IfFileExists{microtype.sty}{% use microtype if available
  \usepackage[]{microtype}
  \UseMicrotypeSet[protrusion]{basicmath} % disable protrusion for tt fonts
}{}
\makeatletter
\@ifundefined{KOMAClassName}{% if non-KOMA class
  \IfFileExists{parskip.sty}{%
    \usepackage{parskip}
  }{% else
    \setlength{\parindent}{0pt}
    \setlength{\parskip}{6pt plus 2pt minus 1pt}}
}{% if KOMA class
  \KOMAoptions{parskip=half}}
\makeatother
\usepackage{xcolor}
\IfFileExists{xurl.sty}{\usepackage{xurl}}{} % add URL line breaks if available
\IfFileExists{bookmark.sty}{\usepackage{bookmark}}{\usepackage{hyperref}}
\hypersetup{
  pdftitle={assignment\_1},
  pdfauthor={Sumayyah Musa},
  hidelinks,
  pdfcreator={LaTeX via pandoc}}
\urlstyle{same} % disable monospaced font for URLs
\usepackage[margin=1in]{geometry}
\usepackage{color}
\usepackage{fancyvrb}
\newcommand{\VerbBar}{|}
\newcommand{\VERB}{\Verb[commandchars=\\\{\}]}
\DefineVerbatimEnvironment{Highlighting}{Verbatim}{commandchars=\\\{\}}
% Add ',fontsize=\small' for more characters per line
\usepackage{framed}
\definecolor{shadecolor}{RGB}{248,248,248}
\newenvironment{Shaded}{\begin{snugshade}}{\end{snugshade}}
\newcommand{\AlertTok}[1]{\textcolor[rgb]{0.94,0.16,0.16}{#1}}
\newcommand{\AnnotationTok}[1]{\textcolor[rgb]{0.56,0.35,0.01}{\textbf{\textit{#1}}}}
\newcommand{\AttributeTok}[1]{\textcolor[rgb]{0.77,0.63,0.00}{#1}}
\newcommand{\BaseNTok}[1]{\textcolor[rgb]{0.00,0.00,0.81}{#1}}
\newcommand{\BuiltInTok}[1]{#1}
\newcommand{\CharTok}[1]{\textcolor[rgb]{0.31,0.60,0.02}{#1}}
\newcommand{\CommentTok}[1]{\textcolor[rgb]{0.56,0.35,0.01}{\textit{#1}}}
\newcommand{\CommentVarTok}[1]{\textcolor[rgb]{0.56,0.35,0.01}{\textbf{\textit{#1}}}}
\newcommand{\ConstantTok}[1]{\textcolor[rgb]{0.00,0.00,0.00}{#1}}
\newcommand{\ControlFlowTok}[1]{\textcolor[rgb]{0.13,0.29,0.53}{\textbf{#1}}}
\newcommand{\DataTypeTok}[1]{\textcolor[rgb]{0.13,0.29,0.53}{#1}}
\newcommand{\DecValTok}[1]{\textcolor[rgb]{0.00,0.00,0.81}{#1}}
\newcommand{\DocumentationTok}[1]{\textcolor[rgb]{0.56,0.35,0.01}{\textbf{\textit{#1}}}}
\newcommand{\ErrorTok}[1]{\textcolor[rgb]{0.64,0.00,0.00}{\textbf{#1}}}
\newcommand{\ExtensionTok}[1]{#1}
\newcommand{\FloatTok}[1]{\textcolor[rgb]{0.00,0.00,0.81}{#1}}
\newcommand{\FunctionTok}[1]{\textcolor[rgb]{0.00,0.00,0.00}{#1}}
\newcommand{\ImportTok}[1]{#1}
\newcommand{\InformationTok}[1]{\textcolor[rgb]{0.56,0.35,0.01}{\textbf{\textit{#1}}}}
\newcommand{\KeywordTok}[1]{\textcolor[rgb]{0.13,0.29,0.53}{\textbf{#1}}}
\newcommand{\NormalTok}[1]{#1}
\newcommand{\OperatorTok}[1]{\textcolor[rgb]{0.81,0.36,0.00}{\textbf{#1}}}
\newcommand{\OtherTok}[1]{\textcolor[rgb]{0.56,0.35,0.01}{#1}}
\newcommand{\PreprocessorTok}[1]{\textcolor[rgb]{0.56,0.35,0.01}{\textit{#1}}}
\newcommand{\RegionMarkerTok}[1]{#1}
\newcommand{\SpecialCharTok}[1]{\textcolor[rgb]{0.00,0.00,0.00}{#1}}
\newcommand{\SpecialStringTok}[1]{\textcolor[rgb]{0.31,0.60,0.02}{#1}}
\newcommand{\StringTok}[1]{\textcolor[rgb]{0.31,0.60,0.02}{#1}}
\newcommand{\VariableTok}[1]{\textcolor[rgb]{0.00,0.00,0.00}{#1}}
\newcommand{\VerbatimStringTok}[1]{\textcolor[rgb]{0.31,0.60,0.02}{#1}}
\newcommand{\WarningTok}[1]{\textcolor[rgb]{0.56,0.35,0.01}{\textbf{\textit{#1}}}}
\usepackage{graphicx,grffile}
\makeatletter
\def\maxwidth{\ifdim\Gin@nat@width>\linewidth\linewidth\else\Gin@nat@width\fi}
\def\maxheight{\ifdim\Gin@nat@height>\textheight\textheight\else\Gin@nat@height\fi}
\makeatother
% Scale images if necessary, so that they will not overflow the page
% margins by default, and it is still possible to overwrite the defaults
% using explicit options in \includegraphics[width, height, ...]{}
\setkeys{Gin}{width=\maxwidth,height=\maxheight,keepaspectratio}
% Set default figure placement to htbp
\makeatletter
\def\fps@figure{htbp}
\makeatother
\setlength{\emergencystretch}{3em} % prevent overfull lines
\providecommand{\tightlist}{%
  \setlength{\itemsep}{0pt}\setlength{\parskip}{0pt}}
\setcounter{secnumdepth}{-\maxdimen} % remove section numbering

\title{assignment\_1}
\author{Sumayyah Musa}
\date{1/20/2021}

\begin{document}
\maketitle

\hypertarget{reading-in-the-simulated-data}{%
\subsection{Reading in the simulated
data}\label{reading-in-the-simulated-data}}

\begin{Shaded}
\begin{Highlighting}[]
\NormalTok{df <-}\StringTok{ }\KeywordTok{read.csv}\NormalTok{(}\StringTok{"simulated_data_explore.csv"}\NormalTok{)}
\KeywordTok{head}\NormalTok{(df, }\DecValTok{5}\NormalTok{)}
\end{Highlighting}
\end{Shaded}

\begin{verbatim}
##   id age_years bench_press_max_lbs height_cm weight_kg
## 1  1  38.62033           122.34489  161.0401  54.07658
## 2  2  35.26406           133.98912  151.7275  24.98297
## 3  3  55.00560            85.90034  160.7133  55.13407
## 4  4  21.77767           216.96535  147.2229  65.85491
## 5  5  49.39280            95.66171  145.1975  67.43110
\end{verbatim}

\hypertarget{changing-the-type-of-variable-id}{%
\subsection{Changing the type of variable
``id''}\label{changing-the-type-of-variable-id}}

\begin{Shaded}
\begin{Highlighting}[]
\NormalTok{df}\OperatorTok{$}\NormalTok{id <-}\StringTok{ }\KeywordTok{as.character}\NormalTok{(df}\OperatorTok{$}\NormalTok{id)}
\end{Highlighting}
\end{Shaded}

\hypertarget{identify-the-variable-types-for-each-variable-in-the-dataset-by-taking-a-glimpse-of-the-data}{%
\subsection{1. Identify the variable types for each variable in the
dataset by taking a glimpse of the
data}\label{identify-the-variable-types-for-each-variable-in-the-dataset-by-taking-a-glimpse-of-the-data}}

\begin{Shaded}
\begin{Highlighting}[]
\KeywordTok{glimpse}\NormalTok{(df)}
\end{Highlighting}
\end{Shaded}

\begin{verbatim}
## Rows: 10,000
## Columns: 5
## $ id                  <chr> "1", "2", "3", "4", "5", "6", "7", "8", "9", "1...
## $ age_years           <dbl> 38.62033, 35.26406, 55.00560, 21.77767, 49.3928...
## $ bench_press_max_lbs <dbl> 122.34489, 133.98912, 85.90034, 216.96535, 95.6...
## $ height_cm           <dbl> 161.0401, 151.7275, 160.7133, 147.2229, 145.197...
## $ weight_kg           <dbl> 54.07658, 24.98297, 55.13407, 65.85491, 67.4311...
\end{verbatim}

\begin{enumerate}
\def\labelenumi{\arabic{enumi}.}
\tightlist
\item
  id \emph{Definition - a variable that identifies each participant.
  }Type - Qualitative, Categorical, Nominal
\item
  age\_years Definition - age of the participants in years. Type -
  Quantitative, Numerical, Continuous
\item
  bench\_press\_max\_lbs Definition - a measure of the maximal weight a
  subject can lift with one repetition, using the bench press exercise,
  in pounds. Type - Quantitative, Numerical, Continuous
\item
  height\_cm Definition - height of the participants in centimeters.
  Type - Quantitative, Numerical, Continuous
\item
  weight\_kg Definition - weight of the participants in kilograms. Type
  - Quantitative, Numerical, Continuous
\end{enumerate}

\hypertarget{calculate-the-bmi-for-each-participant}{%
\subsection{2. Calculate the BMI for each
participant}\label{calculate-the-bmi-for-each-participant}}

Body Mass Index (BMI) is a person's weight in kilograms divided by the
square of height in meters

\begin{Shaded}
\begin{Highlighting}[]
\NormalTok{df <-}\StringTok{ }\NormalTok{df }\OperatorTok\StringTok{ }
\StringTok{  }\KeywordTok{mutate}\NormalTok{(}\DataTypeTok{bmi =}\NormalTok{ weight_kg}\OperatorTok{/}\NormalTok{((height_cm}\OperatorTok{/}\DecValTok{100}\NormalTok{)}\OperatorTok{^}\DecValTok{2}\NormalTok{))}
\end{Highlighting}
\end{Shaded}

\hypertarget{calculate-a-young-and-old-variable-as-per-data-dictionary}{%
\subsection{3. Calculate a young and old variable as per data
dictionary}\label{calculate-a-young-and-old-variable-as-per-data-dictionary}}

\begin{Shaded}
\begin{Highlighting}[]
\NormalTok{df <-}\StringTok{ }\NormalTok{df }\OperatorTok
\StringTok{          }\KeywordTok{mutate}\NormalTok{(}\DataTypeTok{age_category =} \KeywordTok{case_when}\NormalTok{(}
\NormalTok{            age_years }\OperatorTok{<}\StringTok{ }\DecValTok{40} \OperatorTok{~}\StringTok{ "young"}\NormalTok{, }
\NormalTok{            age_years }\OperatorTok{>=}\StringTok{ }\DecValTok{40} \OperatorTok{~}\StringTok{ "old"}\NormalTok{, }
\NormalTok{          ))}
\KeywordTok{glimpse}\NormalTok{(df)}
\end{Highlighting}
\end{Shaded}

\begin{verbatim}
## Rows: 10,000
## Columns: 7
## $ id                  <chr> "1", "2", "3", "4", "5", "6", "7", "8", "9", "1...
## $ age_years           <dbl> 38.62033, 35.26406, 55.00560, 21.77767, 49.3928...
## $ bench_press_max_lbs <dbl> 122.34489, 133.98912, 85.90034, 216.96535, 95.6...
## $ height_cm           <dbl> 161.0401, 151.7275, 160.7133, 147.2229, 145.197...
## $ weight_kg           <dbl> 54.07658, 24.98297, 55.13407, 65.85491, 67.4311...
## $ bmi                 <dbl> 20.85168, 10.85214, 21.34599, 30.38346, 31.9846...
## $ age_category        <chr> "young", "young", "old", "young", "old", "old",...
\end{verbatim}

\hypertarget{calculate-the-mean-and-standard-deviation-for-the-variables-where-it-is-appropriate-including-the-new-variables}{%
\subsection{4. Calculate the mean and standard deviation for the
variables where it is appropriate, including the new
variables}\label{calculate-the-mean-and-standard-deviation-for-the-variables-where-it-is-appropriate-including-the-new-variables}}

\begin{Shaded}
\begin{Highlighting}[]
\KeywordTok{summary}\NormalTok{(df)}
\end{Highlighting}
\end{Shaded}

\begin{verbatim}
##       id              age_years     bench_press_max_lbs   height_cm     
##  Length:10000       Min.   :18.01   Min.   : 55.59      Min.   : 97.19  
##  Class :character   1st Qu.:34.96   1st Qu.: 69.31      1st Qu.:145.41  
##  Mode  :character   Median :51.30   Median : 92.11      Median :154.87  
##                     Mean   :51.52   Mean   :109.13      Mean   :154.86  
##                     3rd Qu.:68.17   3rd Qu.:135.15      3rd Qu.:164.38  
##                     Max.   :84.99   Max.   :262.31      Max.   :211.28  
##    weight_kg            bmi         age_category      
##  Min.   :  9.055   Min.   : 4.221   Length:10000      
##  1st Qu.: 51.703   1st Qu.:21.576   Class :character  
##  Median : 61.990   Median :25.654   Mode  :character  
##  Mean   : 61.906   Mean   :26.027                     
##  3rd Qu.: 72.123   3rd Qu.:29.945                     
##  Max.   :117.922   Max.   :68.838
\end{verbatim}

\#Mean and SD of Age in Years

\begin{Shaded}
\begin{Highlighting}[]
\KeywordTok{mean}\NormalTok{(df}\OperatorTok{$}\NormalTok{age_years)}
\end{Highlighting}
\end{Shaded}

\begin{verbatim}
## [1] 51.51981
\end{verbatim}

\begin{Shaded}
\begin{Highlighting}[]
\KeywordTok{sd}\NormalTok{(df}\OperatorTok{$}\NormalTok{age_years)}
\end{Highlighting}
\end{Shaded}

\begin{verbatim}
## [1] 19.27264
\end{verbatim}

The mean age is 51 years. This sample is not representative of the
population, as it is slightly older than the median age of the resident
population in Newfoundland and Labrador, which is 47.4 years
(\url{https://www.statista.com/statistics/444816/canada-median-age-of-resident-population-by-province/}).

The standard deviation is about 37\% less than the mean age, making the
data values a bit closely spread to the mean, but might not be so
reliable.

\#Mean and SD of Bench Press Max in Pounds

\begin{Shaded}
\begin{Highlighting}[]
\KeywordTok{mean}\NormalTok{(df}\OperatorTok{$}\NormalTok{bench_press_max_lbs)}
\end{Highlighting}
\end{Shaded}

\begin{verbatim}
## [1] 109.134
\end{verbatim}

\begin{Shaded}
\begin{Highlighting}[]
\KeywordTok{sd}\NormalTok{(df}\OperatorTok{$}\NormalTok{bench_press_max_lbs)}
\end{Highlighting}
\end{Shaded}

\begin{verbatim}
## [1] 50.4297
\end{verbatim}

\#Mean and SD of height in Centimeters

\begin{Shaded}
\begin{Highlighting}[]
\KeywordTok{mean}\NormalTok{(df}\OperatorTok{$}\NormalTok{height_cm)}
\end{Highlighting}
\end{Shaded}

\begin{verbatim}
## [1] 154.8626
\end{verbatim}

\begin{Shaded}
\begin{Highlighting}[]
\KeywordTok{sd}\NormalTok{(df}\OperatorTok{$}\NormalTok{height_cm)}
\end{Highlighting}
\end{Shaded}

\begin{verbatim}
## [1] 14.13637
\end{verbatim}

\#Mean and SD of bmi in kg/m\^{}2

\begin{Shaded}
\begin{Highlighting}[]
\KeywordTok{mean}\NormalTok{(df}\OperatorTok{$}\NormalTok{bmi)}
\end{Highlighting}
\end{Shaded}

\begin{verbatim}
## [1] 26.0272
\end{verbatim}

\begin{Shaded}
\begin{Highlighting}[]
\KeywordTok{sd}\NormalTok{(df}\OperatorTok{$}\NormalTok{bmi)}
\end{Highlighting}
\end{Shaded}

\begin{verbatim}
## [1] 6.556188
\end{verbatim}

\#\#5. Calculate the frequencies for the variables where it is
appropriate to do so

\begin{Shaded}
\begin{Highlighting}[]
\KeywordTok{count}\NormalTok{(df,}\StringTok{'age_category'}\NormalTok{)}
\end{Highlighting}
\end{Shaded}

\begin{verbatim}
##   age_category freq
## 1          old 6702
## 2        young 3298
\end{verbatim}

\#\#6. Draw the histogram and discuss the normality of the data

\includegraphics{assignment_1_files/figure-latex/pressure-1.pdf}

Note that the \texttt{echo\ =\ FALSE} parameter was added to the code
chunk to prevent printing of the R code that generated the plot.

\end{document}
